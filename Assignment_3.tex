\documentclass{vhdl-assignment}
\usetikzlibrary{circuits.logic.IEC,calc}

\title{Assignment 3 : Structural Model}
\date{October 2, 2023}

% ==========================================Begin==========================================

\begin{document}
\maketitle
\thispagestyle{fancy}

\begin{problem}{Write testbench for all problems in Assignment 2}
    \begin{enumerate}
        \item Problem 1 
        \item 4-to-1 Multiplexer
        \item Half Adder
        \item Full Adder
        \item 4-bit Ripple Carry Full Adder
    \end{enumerate}
\end{problem}

\begin{problem}{8-to-3 Encoder and 3-to-8 Decoder}
    % Design a 3-to-8 decoder, using a Verilog RTL description. A 3-bit input a[2:0]
    % is provided to the decoder. The output of the decoder is out[7:0]. The output bit
    % indexed by a[2:0] gets the value 1, the other bits are 0. Synthesize the decoder,
    % using any technology library available to you. Optimize for smallest area.
    % Apply identical stimulus to the RTL and the gate-level netlist and compare the
    % outputs.
    \begin{enumerate}
        \item Construct truth table.
        \item Determine output function.
        \item Write Verilog code and testbench for that circuit.
    \end{enumerate}

    \begin{figure}[H]
        \centering
        \begin{subfigure}{0.5\textwidth}
            \centering
            \begin{circuitikz}
                \node[decoder_3_to_8] (Decoder0) {Decoder};
                \node[left] (i0) at (Decoder0.lpin 1) {$i0$};
                \node[left] (i1) at (Decoder0.lpin 2) {$i1$};
                \node[left] (i2) at (Decoder0.lpin 3) {$i2$};
                \node[right] (y0) at (Decoder0.rpin 1) {$y0$};
                \node[right] (y1) at (Decoder0.rpin 2) {$y1$};
                \node[right] (y2) at (Decoder0.rpin 3) {$y2$};
                \node[right] (y3) at (Decoder0.rpin 4) {$y3$};
                \node[right] (y4) at (Decoder0.rpin 5) {$y4$};
                \node[right] (y5) at (Decoder0.rpin 6) {$y5$};
                \node[right] (y6) at (Decoder0.rpin 7) {$y6$};
                \node[right] (y7) at (Decoder0.rpin 8) {$y7$};
            \end{circuitikz}
            \caption{3-to-8 Decoder}
        \end{subfigure}
        \begin{subfigure}{0.4\textwidth}
            \centering
            \begin{circuitikz}
                \node[encoder_8_to_3] (Encoder0) {Encoder};
                \node[left] (i0) at (Encoder0.lpin 1) {$i0$};
                \node[left] (i1) at (Encoder0.lpin 2) {$i1$};
                \node[left] (i2) at (Encoder0.lpin 3) {$i2$};
                \node[left] (i3) at (Encoder0.lpin 4) {$i3$};
                \node[left] (i4) at (Encoder0.lpin 5) {$i4$};
                \node[left] (i5) at (Encoder0.lpin 6) {$i5$};
                \node[left] (i6) at (Encoder0.lpin 7) {$i6$};
                \node[left] (i7) at (Encoder0.lpin 8) {$i7$};
                \node[right] (y0) at (Encoder0.rpin 1) {$y0$};
                \node[right] (y1) at (Encoder0.rpin 2) {$y1$};
                \node[right] (y2) at (Encoder0.rpin 3) {$y2$};
            \end{circuitikz}
            \caption{8-to-3 Encoder}
        \end{subfigure}

        \caption{Encoder and Decoder}
    \end{figure}
\end{problem}

\begin{problem}{Flip Flops : T-FF, D-FF, JK-FF, SR-FF}
    \begin{enumerate}
        \item Construct truth table.
        \item Determine output function.
        \item Write Verilog code and testbench for that circuit.
    \end{enumerate}

    \begin{figure}[H]
        \centering
        \begin{subfigure}{0.2\textwidth}
            \centering
            \begin{circuitikz}
                \node[D_FF]{};
            \end{circuitikz}
        \end{subfigure}
        \begin{subfigure}{0.2\textwidth}
            \centering
            \begin{circuitikz}
                \node[T_FF]{};
            \end{circuitikz}
        \end{subfigure}
        \begin{subfigure}{0.2\textwidth}
            \centering
            \begin{circuitikz}
                \node[SR_FF]{};
            \end{circuitikz}
        \end{subfigure}
        \begin{subfigure}{0.2\textwidth}
            \centering
            \begin{circuitikz}
                \node[JK_FF]{};
            \end{circuitikz}       
        \end{subfigure}
        \caption{Flip Flops}
    \end{figure}
\end{problem}

\note{Using structural model. Flip flops include reset signal.}

\end{document}