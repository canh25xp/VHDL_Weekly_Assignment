\documentclass{vhdl-assignment}

\title{Assignment 2 Solution}
\date{1 Oct 2023}

\begin{document}
\maketitle

\section*{Problem 1}

$F(A,B,C,D)=\sum m(0,2,3,8,9,10,11,12,13,14,15)$

\begin{figure}[H]
    \centering
    % \begin{karnaugh-map}[4][4][1][$Z$][$Y$][$X$][$W$]
    %     \manualterms{0,1,2,3,4,5,6,7,8,9,10,11,12,13,14,15}
    % \end{karnaugh-map}
    \begin{karnaugh-map}[4][4][1][$D$][$C$][$B$][$A$]
        \minterms{0,2,3,8,9,10,11,12,13,14,15}
        \autoterms[0]
        \implicant{12}{10}
        \implicantcorner
        \implicantedge{3}{2}{11}{10}
    \end{karnaugh-map}
    \caption{Karnaugh map}
\end{figure}

$\Rightarrow F=A+B'D'+B'C$

\begin{table}[H]
    \centering
    \begin{displaymath}
        \begin{array}{c c c c|c}
            A & B & C & D & F \\
            \hline
            0 & 0 & 0 & 0 & 1 \\
            0 & 0 & 0 & 1 & 0 \\
            0 & 0 & 1 & 0 & 1 \\
            0 & 0 & 1 & 1 & 1 \\
            0 & 1 & 0 & 0 & 0 \\
            0 & 1 & 0 & 1 & 0 \\
            0 & 1 & 1 & 0 & 0 \\
            0 & 1 & 1 & 1 & 0 \\
            1 & 0 & 0 & 0 & 1 \\
            1 & 0 & 0 & 1 & 1 \\
            1 & 0 & 1 & 0 & 1 \\
            1 & 0 & 1 & 1 & 1 \\
            1 & 1 & 0 & 0 & 1 \\
            1 & 1 & 0 & 1 & 1 \\
            1 & 1 & 1 & 0 & 1 \\
            1 & 1 & 1 & 1 & 1 \\
        \end{array}
    \end{displaymath}
    \caption[short]{Truth Table}
\end{table}

\begin{figure}[H]
    \centering
    \begin{circuitikz}
        \node (A) at (0,0) {$A$};
        \node (B) at (1,0) {$B$};
        \node (C) at (2,0) {$C$};
        \node (D) at (3,0) {$D$};

        \node[not port]                 (Not0)  at ($(D)    + (2, -2)$) {};
        \node[not port]                 (Not1)  at ($(D)    + (2, -5)$) {};
        \node[not port]                 (Not2)  at ($(D)    + (2, -7)$) {};
        \node[and port]                 (And0)  at ($(Not0) + (3, -1)$) {};
        \node[and port]                 (And1)  at ($(And0) + (0, -3)$) {};
        \node[or port, number inputs=3] (Or1)   at ($(And0) + (3, -1)$) {};

        \draw (A) |- ($(A)+(0,-8)$); 
        \draw (B) |- ($(B)+(0,-8)$);
        \draw (C) |- ($(C)+(0,-8)$);
        \draw (D) |- ($(D)+(0,-8)$);

        \filldraw[black] ($(A)+(0,-1)$) circle (2pt) node[]{};
        \draw ($(A)+(0,-1)$) |- ($(A)+(9,-1)$) |- (Or1.in 1);

        \filldraw[black] ($(B)+(0,-2)$) circle (2pt) node[]{};
        \draw ($(B)+(0,-2)$) |- (Not0.in);

        \filldraw[black] ($(B)+(0,-5)$) circle (2pt) node[]{};
        \draw ($(B)+(0,-5)$) |- (Not1.in);

        \filldraw[black] ($(D)+(0,-7)$) circle (2pt) node[]{};
        \draw ($(D)+(0,-7)$) |- (Not2.in);

        \filldraw[black] ($(C)+(0,-3)$) circle (2pt) node[]{};
        \draw ($(C)+(0,-3)$) |- ($(C)+(3,-3)$) |- (And0.in 2);

        \draw (Not0.out) |- (And0.in 1);
        \draw (Not1.out) |- (And1.in 1);
        \draw (Not2.out) |- (And1.in 2);
        \draw (And0.out) |- (Or1.in 2);
        \draw (And1.out) |- (Or1.in 3);

        \node[right] (F) at (Or1.out) {$F$};
    \end{circuitikz}
    \caption{Circuit}
\end{figure}

\begin{lstlisting}
import numpy as np
    
def incmatrix(genl1,genl2):
    m = len(genl1)
    n = len(genl2)
    M = None #to become the incidence matrix
    VT = np.zeros((n*m,1), int)  #dummy variable
    
    #compute the bitwise xor matrix
    M1 = bitxormatrix(genl1)
    M2 = np.triu(bitxormatrix(genl2),1) 

    for i in range(m-1):
        for j in range(i+1, m):
            [r,c] = np.where(M2 == M1[i,j])
            for k in range(len(r)):
                VT[(i)*n + r[k]] = 1;
                VT[(i)*n + c[k]] = 1;
                VT[(j)*n + r[k]] = 1;
                VT[(j)*n + c[k]] = 1;
                
                if M is None:
                    M = np.copy(VT)
                else:
                    M = np.concatenate((M, VT), 1)
                
                VT = np.zeros((n*m,1), int)
    
    return M
\end{lstlisting}

\pagebreak
\section*{Promblem 2 : Design a 4-to-1 Multiplexer }

\begin{enumerate}
    \item Construct truth table.
    \item Determine output function.
    \item Draw the circuit.
    \item Design a \emph{Structural module} for this circuit (Using Verilog)
\end{enumerate}

\section*{Promblem 3 : Adder Circuit}

For each of the following circuit, Construct truth table, Determine the output function and Write Verilog code 

\begin{enumerate}
    \item Half Adder
    \item Full Adder
    \item Ripple Carry 4-bit Adder
    \item Ripple Carry 16-bit Adder
\end{enumerate}

\end{document}