\documentclass{vhdl-assignment}
\usetikzlibrary{circuits.logic.IEC,calc}

\title{Assignment 3 : Structural Model}
\date{October 8, 2023}

\begin{document}
\maketitle
\thispagestyle{fancy}

\begin{problem}{Write testbench for all problems in Assignment 2}
    \subsection*{1. Problem 1}
    \verilog[Problem 1 testbench]{src/Assignment_1_Problem_1_tb.v}
    Notice how the input I is defined to be 5 bits wide (line 2).
    This is to prevent the for loop (line 6) looping infinitely.
    Because if I is only 4bits, the variable itself never exceed 15.
    This is completely optional, alternatively, you can manually break the simulation at a desire time point,
    by typing the command "\verb|run 100|" in the transcript command windows (100 is the duration you want to run, change it to suit the simulation) 
    
    The above testbench should output something like following:
\begin{lstlisting}[caption=Problems 1 Testbench Output]
#                    5 I=00000  F=1
#                   10 I=00001  F=1
#                   15 I=00010  F=0
#                   20 I=00011  F=0
#                   25 I=00100  F=1
#                   30 I=00101  F=1
#                   35 I=00110  F=1
#                   40 I=00111  F=1
#                   45 I=01000  F=0
#                   50 I=01001  F=0
#                   55 I=01010  F=0
#                   60 I=01011  F=0
#                   65 I=01100  F=0
#                   70 I=01101  F=0
#                   75 I=01110  F=0
#                   80 I=01111  F=0
\end{lstlisting}

    \subsection*{2. 4-to-1 Multiplexer}
    \verilog[4-to-1 Multiplexer testbench]{src/Assignment_1_Problem_2_tb.v}
\begin{lstlisting}[caption=4-to-1 Multiplexer Testbench Output]
#                    5 IN0= 1, IN1= 0, IN2= 1, IN3= 0
#                   10 S1 = 0, S0 = 0, OUTPUT = 1
#                   15 S1 = 0, S0 = 1, OUTPUT = 0
#                   20 S1 = 1, S0 = 0, OUTPUT = 1
#                   25 S1 = 1, S0 = 1, OUTPUT = 0
\end{lstlisting}

    \pagebreak
    \subsection*{3. Half Adder}
    \verilog[Half Adder testbench]{src/Assignment_1_Problem_3_Half_Adder_tb.v}
\begin{lstlisting}[caption=Half Adder Testbench Output]
#                    0 X= 0, Y=0, S= 0, C= 0
#                    5 X= 0, Y=1, S= 1, C= 0
#                   10 X= 1, Y=0, S= 1, C= 0
#                   15 X= 1, Y=1, S= 0, C= 1
\end{lstlisting}

    \subsection*{4. Full Adder}
    \verilog[Full Adder testbench]{src/Assignment_1_Problem_3_Full_Adder_tb.v}
\begin{lstlisting}[caption=Full Adder Testbench Output]
#                    0 X= 0, Y=0, Ci= 0, S= 0, Co= 0
#                    5 X= 0, Y=0, Ci= 1, S= 1, Co= 0
#                   10 X= 0, Y=1, Ci= 0, S= 1, Co= 0
#                   15 X= 0, Y=1, Ci= 1, S= 0, Co= 1
#                   20 X= 1, Y=0, Ci= 0, S= 1, Co= 0
#                   25 X= 1, Y=0, Ci= 1, S= 0, Co= 1
#                   30 X= 1, Y=1, Ci= 0, S= 0, Co= 1
#                   35 X= 1, Y=1, Ci= 1, S= 1, Co= 1    
\end{lstlisting}
    
    \subsection*{5. 4-bit Ripple Carry Adder}
    \verilog[4-bit Ripple Carry Adder testbench]{src/Assignment_1_Problem_3_Ripple_Carry_Adder_tb.v}
\begin{lstlisting}[caption=Full Adder Testbench Output]
#                    0 X= 0000, Y=0000, Ci= 0, S= 0000, Co= 0
#                    5 X= 0011, Y=0100, Ci= 0, S= 0111, Co= 0
#                   10 X= 0010, Y=0101, Ci= 0, S= 0111, Co= 0
#                   15 X= 1001, Y=1001, Ci= 0, S= 0010, Co= 1
#                   20 X= 1010, Y=1111, Ci= 0, S= 1001, Co= 1
#                   25 X= 1010, Y=0101, Ci= 1, S= 0000, Co= 1   
\end{lstlisting}
\end{problem}

\newpage
\begin{problem}{8-to-3 Encoder and 3-to-8 Decoder}
    \begin{enumerate}
        \item Construct truth table.
        \item Determine output function.
        \item Write Verilog code and testbench for that circuit.
    \end{enumerate}

    \begin{figure}[H]
        \centering
        \begin{subfigure}{0.5\textwidth}
            \centering
            \begin{circuitikz}
                \node[decoder_3_to_8] (Decoder0) {Decoder};
                \node[left] (i0) at (Decoder0.lpin 1) {$i0$};
                \node[left] (i1) at (Decoder0.lpin 2) {$i1$};
                \node[left] (i2) at (Decoder0.lpin 3) {$i2$};
                \node[right] (y0) at (Decoder0.rpin 1) {$y0$};
                \node[right] (y1) at (Decoder0.rpin 2) {$y1$};
                \node[right] (y2) at (Decoder0.rpin 3) {$y2$};
                \node[right] (y3) at (Decoder0.rpin 4) {$y3$};
                \node[right] (y4) at (Decoder0.rpin 5) {$y4$};
                \node[right] (y5) at (Decoder0.rpin 6) {$y5$};
                \node[right] (y6) at (Decoder0.rpin 7) {$y6$};
                \node[right] (y7) at (Decoder0.rpin 8) {$y7$};
            \end{circuitikz}
            \caption{3-to-8 Decoder}
        \end{subfigure}
        \begin{subfigure}{0.4\textwidth}
            \centering
            \begin{circuitikz}
                \node[encoder_8_to_3] (Encoder0) {Encoder};
                \node[left] (i0) at (Encoder0.lpin 1) {$i0$};
                \node[left] (i1) at (Encoder0.lpin 2) {$i1$};
                \node[left] (i2) at (Encoder0.lpin 3) {$i2$};
                \node[left] (i3) at (Encoder0.lpin 4) {$i3$};
                \node[left] (i4) at (Encoder0.lpin 5) {$i4$};
                \node[left] (i5) at (Encoder0.lpin 6) {$i5$};
                \node[left] (i6) at (Encoder0.lpin 7) {$i6$};
                \node[left] (i7) at (Encoder0.lpin 8) {$i7$};
                \node[right] (y0) at (Encoder0.rpin 1) {$y0$};
                \node[right] (y1) at (Encoder0.rpin 2) {$y1$};
                \node[right] (y2) at (Encoder0.rpin 3) {$y2$};
            \end{circuitikz}
            \caption{8-to-3 Encoder}
        \end{subfigure}

        \caption{Encoder and Decoder}
    \end{figure}

    \subsection*{a. Decoder}
    \begin{table}[H]
        \begin{displaymath}
            \begin{array}{c c c|c c c c c c c c}
                i0 & i1 & i2 & y0 & y1 & y2 & y3 & y4 & y5 & y6 & y7 \\
                \hline
                0 & 0 & 0 & 1 & 0 & 0 & 0 & 0 & 0 & 0 & 0 \\
                0 & 0 & 1 & 0 & 1 & 0 & 0 & 0 & 0 & 0 & 0 \\
                0 & 1 & 0 & 0 & 0 & 1 & 0 & 0 & 0 & 0 & 0 \\
                0 & 1 & 1 & 0 & 0 & 0 & 1 & 0 & 0 & 0 & 0 \\
                1 & 0 & 0 & 0 & 0 & 0 & 0 & 1 & 0 & 0 & 0 \\
                1 & 0 & 1 & 0 & 0 & 0 & 0 & 0 & 1 & 0 & 0 \\
                1 & 1 & 0 & 0 & 0 & 0 & 0 & 0 & 0 & 1 & 0 \\
                1 & 1 & 1 & 0 & 0 & 0 & 0 & 0 & 0 & 0 & 1 \\
            \end{array}
        \end{displaymath}
        \caption[short]{Decoder 3-to-8 Truth Table}
    \end{table}
    \begin{equation*}
        \Rightarrow
        \begin{cases}
            y0=i0'i1'i2' \\
            y1=i0'i1'i2 \\
            y2=i0'i1i2' \\
            y3=i0'i1i2 \\
            y4=i0i1'i2' \\
            y5=i0i1'i2 \\
            y6=i0i1i2' \\
            y7=i0i1i2 \\
        \end{cases}
    \end{equation*}

    \verilog[Decoder 3-to-8 and testbench]{src/Decoder_3_to_8.v}
    \begin{lstlisting}[caption=Full Adder Testbench Output]
    #
    #
    #
    #
    #
    #   
    \end{lstlisting}
    \subsection*{b. Encoder}


\end{problem}

\newpage
\begin{problem}{Flip-flops : D-FF, T-FF, JK-FF, SR-FF}
    \begin{enumerate}
        \item Construct truth table.
        \item Determine output function.
        \item Write Verilog code and testbench for that circuit.
    \end{enumerate}

    \begin{figure}[H]
        \centering
        \begin{subfigure}{0.2\textwidth}
            \centering
            \begin{circuitikz}
                \node[D_FF]{};
            \end{circuitikz}
            \caption{D Flip-flop}
        \end{subfigure}
        \begin{subfigure}{0.2\textwidth}
            \centering
            \begin{circuitikz}
                \node[T_FF]{};
            \end{circuitikz}
            \caption{T Flip-flop}
        \end{subfigure}
        \begin{subfigure}{0.2\textwidth}
            \centering
            \begin{circuitikz}
                \node[SR_FF]{};
            \end{circuitikz}
            \caption{SR Flip-flop}
        \end{subfigure}
        \begin{subfigure}{0.2\textwidth}
            \centering
            \begin{circuitikz}
                \node[JK_FF]{};
            \end{circuitikz}     
            \caption{JK Flip-flop}
        \end{subfigure}
        \caption{Flip-flops}
    \end{figure}
\end{problem}

\note{Using structural model. Flip flops include reset signal.}

\end{document}